\documentclass[a4paper,12pt,swedish]{article}

\usepackage[swedish]{babel}
\usepackage[T1]{fontenc}
\usepackage[latin1]{inputenc}
\usepackage{graphics}

\title{INDA final project - A bandwidth-limiting SOCKS 4/5 proxy}
\author{Bj�rn Jensen-Urstad (810814-0271)\\bju@stacken.kth.se}
\date{}


\begin{document}
\maketitle\thispagestyle{empty}

\newpage

\section{Introduction}
Many small private networks often use something called a firewall.
A firewall is a system that isolates the internal network from the
external (which very often is the Internet).

The firewall (often) allows the internal computers access to the outside
network by offering NAT services that operate on the network-layer
or by offering application-layer gateways such as Squid.

SOCKS is a proxy protocol that operates on the application-level and
allows users on the LAN transparent access to the external network
without needing direct IP access. SOCKS is fairly well supported
and can be used with most client software.

\section{Intended user}
The intended user for this software should be familiar with UNIX (the target
platform) and basic networking concepts. He/She is most likely a system-
administrator on a small LAN consisting of between 10 and 100 computers.

His/Her goal is to connect the LAN to the internet in a secure and simple way
and/or limit the bandwidth used by certain applications.

\section{Reasons for using this program}
This applications main feature is it's bandwidth limiting capabilities.
This function is very useful when one for example is creating a website
and want to see how it performs for people with slower internet connections
or when one is mirroring a huge amount of data and want to limit the
``data-flow'' during daytime.

This program can ofcourse also be used as a ``normal'' SOCKS4/5 server.

Another reason for using this program is that is's free.

\section{How will it work}
The program will run on the firewall and listen for connections on
the internal network interface. Once the request is parsed,
``approved'' and the nessesary connections is setup the program will 
relay data between the internal client and the external server.
Since all data has to pass through the program the program can
relay the data in a user-defined speed thus slowing down the
connection.

The program will be a complete SOCKS4/5 implementation as specified in
``RFC 1928'' and in ``SOCKS: A protocol for TCP proxy across firewalls''
by Ying-Da Lee.

\section{Requirements Analysis}
The program will be a SOCKS 4/5 server with the ability to set maximum speed 
in bytes/second for TCP connections passing through based on the day, time of day and the
destination port.

The program will also hava a built in HTTP server for monitoring connections.

Since the program is a daemon it wont have any interactive user-interface.
When the program is started it will print any errors encounted
in the init phase to the console. After the init phase is finished
any errors encountered will be directed to the logfile.

All features will be accessed through the configuration file which
can be reloaded on the fly by sending the program the SIGHUP signal.

\section{Program Design}
This program will be implemented in C++ since that is the language I
know best. The design will be object-oriented.

The configuration details are represented by the config class.
Most of the configuration datails are primitive data types except
the throttling details which is saved in std::vector which
has linear search time.

A socks request is represented by the socks\_request class.
The socks\_request class consists only of primitive types.

A socks reply is represented by the socks\_reply class.
The socks\_reply class consists only of primitive types.

\subsection{IO models}
The most important decision when designing this program is probably
choosing the appropriate IO model for handling the concurrent
network connections. I have chosen to use nonblocking IO in this
program. Each client connection has it's own data structure that
contains client/server sockets and read/write buffers.

Since we has to handle both socks and http connections socks connections
will be represented by the socks\_connection class and http connections will
be represented by the http\_connection class. Both classes will be subclasses
of the connection class and have a common interface.

All connections are then kept in a std::vector which has linear search time.

Review of IO models:
\subsubsection{blocking IO, fork on each request}
This model is very easy to implement and it is the classical way servers are
written in UNIX. The downside is that it's very slow.

\subsubsection{blocking IO, thread for each request}
Instead of forking on each request we can create a new thread instead.
This is probably faster than forking but if fork is not fast enough
this is probably not a good solution. We could create a thread pool,
that way we dont have to create and destroy lots of threads. The program
will still be limited by the number of threads the OS can handle.

\subsubsection{nonblocking IO, IO multiplexing}
In this model we only have one thread that uses select or poll to monitor
readability/writeability on a set of ``file descriptors''. This model is very fast
and probably ideal for relaying data between the connections.
The downside is that it is pretty hard to implement complex servers with
this model. It can ofcourse be done, the best solution is probably by
using a state variable for each client connection but we must still be able
to handle short reads/writes nonblocking connects/accepts etc.

\subsubsection{async IO}
A ``new'' model. I have never used it myself but it's supposed to be very fast.
Sadly it's not widely supported and according to people who have used it
it's hard to use, or as someone said ``like programming with lots of gotos''.

\section{Implementation Plan}
The tasks are implemented in the order they are specified in.
I expect that it will take about a week to implement all of them.

\subsection{task 1 - socket library}
This library must support nonblocking connects/accepts and nonblocking reads and writes.
It must also include a ``serversocket'' class that supports nonblocking accepts
and a select class that can be used to monitor a set of sockets.
The socket and ``serversocket'' classes will be implemented as subclasses of
the file\_descriptor class that select operates on.

This part will be separated from the project and compiled as a standalone
library that can be reused for other projects. It will live in the io:: namespace.

Expected time needed for implementation = 2 days.

\subsection{task 2 - socks4/5 request parser}
The request parser will take a sequence of bytes and try to extract the socks request.
If the input differs from what is specified in the RFC it will notify the user (of the code).
Once the request is parsed it will be saved in a structure.
The parser will be implemented as a standalone function in the socks:: namespace.

Expected time needed for implementation = 4 hours.

\subsection{task 3 - socks4/5 response creator}
The response creator will take the relevant options as arguments and create a valid socks4/5
response. It will be implemented as a single function in the socks:: namespace.

Expected time needed for implementation = 4 hours.

\subsection{task 4 - configuration file parser}
All configuration details are stored in the config class which also parses the configuration file.
The configuration file is a standard text file which can easily be edited.
The options have the form $argument = value$, which is easy to parse.

Expected time needed for implementation = 4 hours.

\subsection{task 5 - logging facility}
All logging is handled by the log class. The log class can direct it's output to the
syslog, a separate file or to the console.

Expected time needed for implementation = 2 hours.

\subsection{task 6 - ``main program logic''}
Here we will bind all previous code together. We will basically monitor all our
sockets for any changes and respond to them in a correct way.
I expect this to be the hardest part so it will probably take at least 3
days to complete.

\section{Test Plan and Summary of Results}
The tests are automated and will test the basic functionality
on a running socks server. The test suites are written in Java.

All tests were successful except the udp relay test since that feature
has not yet been fully implemented. During testing I found several
bugs in the conversation from host to network byte ordering.

The test plan consists of the following tests:

\subsection{socks version 4 connect}
Client binds a serversocket. Client then opens a connection to
the socks server and instruct the server to connect to the previously
bound socket.

\subsection{socks version 4 bind}
Client opens a connection to the socks server and instruct the server to
bind a port. The Client then creates a new socket connected to the bound port.

\subsection{socks version 5 connect}
Client binds a serversocket. Client then opens a connection to
the socks server and instruct the server to connect to the previously
bound port.

\subsection{socks version 5 bind}
Client opens a connection to the socks server and instruct the server to
bind a port. The Client then creates a new socket connected to the bound port.

\subsection{socks version 5 udp relay}
Client opens a UDP serversocket. Client then opens a connection to the
socks server and instruct the server to bind a UDP serversocket and relay
packets between the the two UDP sockets.

\section{User manual}
Socks is a SOCKS 4/5 server with bandwidth limiting capabilities.
It's used to connect a small network to the internet and/or limit 
bandwidth used by certain services.

\subsection{Supported platforms}
Socks has been tested and is known to work under the following platforms:
\begin{itemize}
\item Slackware Linux 2.4 with g++ version 3.2.2.
\item Solaris 8.0 with g++ version 3.?
\item OpenBSD 3.4 with g++ version 3.?
\end{itemize}
Pretty much any new system that is POSIX compatible should work without too
much trouble.

\subsection{Installation}
Unpack socks and enter the socks/src directory. Type make and select
operating system. Once the compilation finishes move the socks binary to
where you want it (often /usr/local/sbin). Don't forget to also move the
socks.conf configuration file to your favourite configuratution directory
(often /usr/local/etc).

\subsection{Running socks}
Befory you start socks open the configuration file (/usr/local/etc/socks.conf)
in your editor and make the changes you want. The configuration file is
commented and pretty straightforward. The only thing that is really important
is to set the listen interfaces correctly since a misconfiguration might let
anyone relay connections through the program.

To start socks simply type: socks [full path to socks.conf].

\section*{}

\begin{tabular}{@{}l}
  Bj�rn Jensen-Urstad\\
  Skyttevagen 48\\
  18146 Liding�\\
  070-2584922\\
  bju@stacken.kth.se\\
\end{tabular}

\end{document}